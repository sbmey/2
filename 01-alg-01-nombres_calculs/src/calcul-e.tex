% !TeX root = ../../../../courant/2-02-alg-2-calculs.tex
% !TeX program = LuaLaTeX-SE
% !TeX encoding = UTF-8
% !TeX spellcheck = fr_FR
%
\begin{center}
\bfseries \Large 
Ces exercices se font sans calculatrice
\end{center}
%
%--<<--<<--<<--<<--<<--<<--<<--<<--<<--<<--<<--<<--<<--<<--<<--<<--<<--<<--<<--<<--<<--<<--
%-->>-->>-->>-->>-->>-->>-->>-->>-->>-->>-->>-->>-->>-->>-->>-->>-->>-->>-->>-->>-->>-->>--
%
\begin{exr} Simplifier les nombres suivants (il faut les écrire sous forme de fractions irréductibles ou de nombres entiers relatifs).

$A=\dfrac1{\frac52}$, \hfill
$B=\dfrac{\frac{14}{27}}{\frac{7}{9}}$, \hfill
$C=\mleft(\dfrac{\frac23}{4}\mright)^2-\mleft(\dfrac{8}{\frac34}\mright)^2$, \hfill
$D=\dfrac{\frac7{12}-\frac53}{\frac52+\frac83}$, \hfill
$E=\mleft(4+\dfrac{5}{7}\mright)\times\mleft(5+\dfrac{4}{9}\mright)$, \hfill
$F=\mleft(1+\frac12\mright)\mleft(1+\frac13\mright)\cdots\mleft(1+\frac19\mright)$

%A=\dfrac{5}{2}+\dfrac{8}{3} 
%B=\dfrac{7}{12}-\dfrac{5}{3} 
%E=\dfrac{5}{7}\times \dfrac{4}{15} 
%F=\dfrac{ \dfrac{8}{9}}{ \dfrac{3}{5}}
%a=\dfrac{1}{ \dfrac{2}{5}} 
%b=\dfrac{4}{ \dfrac{2}{6}}
%c=\dfrac{ \dfrac{5}{2}}{ \dfrac{10}{6}}\ 
%d=\dfrac{ \dfrac{7}{9}}{ \dfrac{14}{27}} 
\end{exr}
%
%--<<--<<--<<--<<--<<--<<--<<--<<--<<--<<--<<--<<--<<--<<--<<--<<--<<--<<--<<--<<--<<--<<--
%-->>-->>-->>-->>-->>-->>-->>-->>-->>-->>-->>-->>-->>-->>-->>-->>-->>-->>-->>-->>-->>-->>--
%
\begin{exr} Simplifier les nombres suivantes.

%$a=\dfrac{(0,001)^3 \times(-10000)^5}{(0,01)^{-4}}$, \hfill
$a=\dfrac{(0,0001)^{-4}\times(10000)^5\times(-0,001)^7}{(10\times  0,01^3)^4}$, \hfill
$b=\dfrac{(-16)^{-4}\times  3^{21}}{6^3\times  9^7}$, \hfill
$c=\dfrac{2^{-5}\times  (-6)^3\times  3^{-4}}{-9^{-2}\times  8^{-4}}$, \hfill
%        \item $C=\dfrac{(-1,2)^4\times(-2)^{13}\times(-1)^2}{(-5)^6\times(-7)^{21}}$ 
$d=7\sqrt{75}-2\sqrt{12}$, \hfill
$e=2\sqrt{5}+\sqrt{0,0045}$, \hfill
$f=\left( 11\sqrt{5}-5\sqrt{11}\right)\left( 11\sqrt{5}+5\sqrt{11}\right)$, \hfill
$g=\dfrac{14}{3\sqrt{7}}$, \hfill
$h=\dfrac{1}{2-\sqrt2}-\dfrac{1}{2+\sqrt2}$, \hfill
$i=\dfrac{2-\sqrt{10}}{3-\sqrt{10}}$.
$j=\dfrac{3}{4\sqrt{2}-3}$.
%$a=\dfrac{7}{2\sqrt{3}}$ 
%$b=\dfrac{1-\sqrt2}{\sqrt2-\sqrt3}$
%$e=\dfrac{3}{\sqrt{2}+\sqrt{3}}$
%f=\dfrac{\sqrt{2}-\sqrt{5}}{\sqrt{2}+\sqrt{5}}$
%$g=\dfrac{2}{4-\sqrt2}\ ; \quad \ 
%$i=\dfrac{2+\sqrt3}{2-\sqrt3}$
%$l=\dfrac{1}{\sqrt2-2}+\dfrac{3}{\sqrt3}$
\end{exr}
%
%--<<--<<--<<--<<--<<--<<--<<--<<--<<--<<--<<--<<--<<--<<--<<--<<--<<--<<--<<--<<--<<--<<--
%-->>-->>-->>-->>-->>-->>-->>-->>-->>-->>-->>-->>-->>-->>-->>-->>-->>-->>-->>-->>-->>-->>--
%
%\begin{exr} 
%Soit $A=\sqrt{9-4\sqrt{5}}+\sqrt{9+4\sqrt{5}}$. Montrer que $A^2 \in \N$.
%\end{exr}
%
%--<<--<<--<<--<<--<<--<<--<<--<<--<<--<<--<<--<<--<<--<<--<<--<<--<<--<<--<<--<<--<<--<<--
%-->>-->>-->>-->>-->>-->>-->>-->>-->>-->>-->>-->>-->>-->>-->>-->>-->>-->>-->>-->>-->>-->>--
%
\begin{exr} 
Soit $n\in \N$. Montrer que $\sqrt{n+1}+\sqrt{n}$ et $\sqrt{n+1}-\sqrt{n}$ sont inverses l'un de l'autre.
\end{exr}
%
%--<<--<<--<<--<<--<<--<<--<<--<<--<<--<<--<<--<<--<<--<<--<<--<<--<<--<<--<<--<<--<<--<<--
%-->>-->>-->>-->>-->>-->>-->>-->>-->>-->>-->>-->>-->>-->>-->>-->>-->>-->>-->>-->>-->>-->>--
%
%\begin{exr}
%Soit $X=\sqrt{10-\sqrt{84}}+\sqrt{10+\sqrt{84}}$.  
%\begin{enumerate}
%\item Calculer $X$ à la calculatrice. 
%\item Développer $X^2$, puis en déduire $X$, et retrouver le résultat
%  précédent.  
%\item Mêmes questions avec $Y=\sqrt{3+\sqrt{5}}-\sqrt{3-\sqrt{5}}$ 
%  et $Z=\sqrt{15-\sqrt{216}}+\sqrt{15+\sqrt{216}}$.
%\end{enumerate}
%\end{exr}
%
%--<<--<<--<<--<<--<<--<<--<<--<<--<<--<<--<<--<<--<<--<<--<<--<<--<<--<<--<<--<<--<<--<<--
%-->>-->>-->>-->>-->>-->>-->>-->>-->>-->>-->>-->>-->>-->>-->>-->>-->>-->>-->>-->>-->>-->>--
%
%\begin{exr}
%\begin{enumerate}
%\item Soit $X=\sqrt{24}-\sqrt{6}$. 
%  Calculer $X^2$, puis en déduire la valeur de $X$. 
%\item Soit $X=\sqrt{50}-\sqrt{8}$. 
%  Calculer $X^2$, puis en déduire la valeur de $X$. 
%\end{enumerate}
%\end{exr}
%
%--<<--<<--<<--<<--<<--<<--<<--<<--<<--<<--<<--<<--<<--<<--<<--<<--<<--<<--<<--<<--<<--<<--
%-->>-->>-->>-->>-->>-->>-->>-->>-->>-->>-->>-->>-->>-->>-->>-->>-->>-->>-->>-->>-->>-->>--
%
%\begin{exr}
%On considère les expressions suivantes : \\[.4em]
%  $\bullet\ A(x)=(x+2)(2x-3)$ \quad  
%  $\bullet B(x)=(3-2x)(3x-2)$ \quad 
%  $\bullet\ C(x)=(x+2)(2x-3)(-3x+1)$\\[.4em]
%  $\bullet\ D(x)=(x+3)^2$ \quad 
%  $\bullet\ E(x)=(3x-4)^2$ \quad 
%  $\bullet\ F(x)=\left( 3x+\dfrac13\right)^2$
%  
%  \begin{tabular}{*4{p{4.2cm}}}
%    $\bullet\ A(2)=\ \dots$
%    & $\bullet\ A(-2)=\ \dots$ 
%    & $\bullet\ B(1)=\ \dots$ 
%    & $\bullet\ B(-1)=\ \dots$ \\[.6em]
%     $\bullet\ C(2)=\ \dots$ 
%    & $\bullet\ C\left( \dfrac{1}{3}\right)=\ \dots$
%    & $\bullet\ D(-1)=\ \dots$ 
%    & $\bullet\ E(-2)=\ \dots$
%  \end{tabular}
%\end{exr}
%
%--<<--<<--<<--<<--<<--<<--<<--<<--<<--<<--<<--<<--<<--<<--<<--<<--<<--<<--<<--<<--<<--<<--
%-->>-->>-->>-->>-->>-->>-->>-->>-->>-->>-->>-->>-->>-->>-->>-->>-->>-->>-->>-->>-->>-->>--
%
%\begin{exr} On sait que $b^3=5,832$ et $b^5=18,89$. 
%    \begin{enumerate}
%    \item Sans calculer $b$, calculer $b^2$ et $b^6$. 
%    \item En déduire~$b$. 
%    \end{enumerate}
%\end{exr}
%
%--<<--<<--<<--<<--<<--<<--<<--<<--<<--<<--<<--<<--<<--<<--<<--<<--<<--<<--<<--<<--<<--<<--
%-->>-->>-->>-->>-->>-->>-->>-->>-->>-->>-->>-->>-->>-->>-->>-->>-->>-->>-->>-->>-->>-->>--
%
\begin{exr}
    \begin{enumerate}
    \item Développer les expressions suivantes.
        \begin{itemize-}[label={}](2)
        \item $A(x)=(6-5x)(3x-4)$
        \item $B(x)=(3-x)^2-(4x+5)^2$ 
        \item $C(x)=(x-1)(2x+1)(-3x+2)$
        \item $D(x)=\mleft(\dfrac{2}{3}x+\dfrac{3x}{4}\mright)\mleft(\dfrac{2x}{3}-\dfrac{3}{4}x\mright)$
        \end{itemize-}
    \item Factoriser les expressions :
        \begin{itemize-}[label={},after-item-skip=-\parskip](2)
        \item $E(x)=(2x-3)(4x-2)-(2x-3)(2x+4)$
        \item $F(x)=9x^2-30x+25$
        \item $G(x)=(x-6)(5x^2+3x)-(10x+6)(x+7)$%
        \item $H(x)=36x^2-49$
        \end{itemize-}
    \item Résoudre les équations.
        \begin{enumerate-}[label=\alph*.](3)
        \item $A(x)=0$
        \item $B(x)=0$
        \item $3(x-1)^2=5(x-1)$
%        \item $H(x)=0$
%        \item $A(x)=B(x)$ 
        \end{enumerate-}
    \end{enumerate}
\end{exr}
%
%--<<--<<--<<--<<--<<--<<--<<--<<--<<--<<--<<--<<--<<--<<--<<--<<--<<--<<--<<--<<--<<--<<--
%-->>-->>-->>-->>-->>-->>-->>-->>-->>-->>-->>-->>-->>-->>-->>-->>-->>-->>-->>-->>-->>-->>--
%
\begin{exr} Simplifier les expressions.
    \begin{itemize-}[label={}](4)
    \item $A=\dfrac{x}{x+1}+\dfrac{2}{5x}$
    \item $B=5+\dfrac{x+4}{x+2}$
    \item $C=x-1+\dfrac{3}{x+3}$
    \item $D=\dfrac{1}{2-3x}-\dfrac{1}{2+3x}$
    \end{itemize-}
\end{exr}
%
%--<<--<<--<<--<<--<<--<<--<<--<<--<<--<<--<<--<<--<<--<<--<<--<<--<<--<<--<<--<<--<<--<<--
%-->>-->>-->>-->>-->>-->>-->>-->>-->>-->>-->>-->>-->>-->>-->>-->>-->>-->>-->>-->>-->>-->>--
%
\enlargethispage{\baselineskip}
\begin{exr}
    \begin{enumerate}
    \item \'Ecrire sans valeur absolue : $A=3\times\abs{-5}-\abs{3\pi-9}$ et $B=\abs{10^{10}-10^{20}}$.
    \item Interpréter les équations en terme de distance avant de les résoudre en utilisant la droite numérique :
        \begin{enumerate-}[label=\alph*.](2)
        \item $\abs{x-2}=4$
        \item $\abs{2+3x}=5$
        \end{enumerate-}
    \item Résolvez (algébriquement) les équations suivantes :
        \begin{enumerate-}[label=\alph*.](3)
        \item $\abs{3x+9}=12$
        \item $\abs{x^2-16}=20$
        \item $\abs{2x-5}=\abs{4x-3}$
        \end{enumerate-}
    \item Résoudre les inéquations suivantes :
        \begin{enumerate-}[label=\alph*.](2)
        \item $\abs{5-2x}<12$
        \item $\abs{4x-9}\geqslant20$
        \end{enumerate-}
\end{enumerate}
\end{exr}


\begin{exr}
\'Ecrire sous forme d'intervalle l'ensemble des réels défini par :
    \begin{enumerate-}(3)
    \item $\abs{x-12}\leqslant 5$
    \item $\abs{x+1}< 3$
    \item $\abs{-2+x}> 6$
    \end{enumerate-}
\end{exr}
%
%--<<--<<--<<--<<--<<--<<--<<--<<--<<--<<--<<--<<--<<--<<--<<--<<--<<--<<--<<--<<--<<--<<--
%-->>-->>-->>-->>-->>-->>-->>-->>-->>-->>-->>-->>-->>-->>-->>-->>-->>-->>-->>-->>-->>-->>--
%
%\begin{exr} Ecrire sous forme de fraction irréductible les nombres: 
%$a=\dfrac{2}{ \dfrac{x+1}{3}}$ ;\ 
%$b=\dfrac{ \dfrac{x^2+x}{x-3}}{ \dfrac{x+1}{x^2-9}}$ ; 
%$c=2+\dfrac{ \dfrac{1}{3}x}{x+1}$
%\end{exr}
%
%--<<--<<--<<--<<--<<--<<--<<--<<--<<--<<--<<--<<--<<--<<--<<--<<--<<--<<--<<--<<--<<--<<--
%-->>-->>-->>-->>-->>-->>-->>-->>-->>-->>-->>-->>-->>-->>-->>-->>-->>-->>-->>-->>-->>-->>--
%
%\begin{exr} Ecrire les nombres suivants sous forme d'une seule fraction 
%sans radicaux au dénominateur:  
%$a=\dfrac{\sqrt3x}{\sqrt2}-\dfrac{\sqrt2x}{\sqrt3}$ ;
%$b=1-\dfrac{3}{\sqrt{x}+2}$;
%$c=\dfrac{\sqrt{x+1}+\sqrt{x}}{\sqrt{x+1}-\sqrt{x}}$
%\end{exr}
%
%--<<--<<--<<--<<--<<--<<--<<--<<--<<--<<--<<--<<--<<--<<--<<--<<--<<--<<--<<--<<--<<--<<--
%-->>-->>-->>-->>-->>-->>-->>-->>-->>-->>-->>-->>-->>-->>-->>-->>-->>-->>-->>-->>-->>-->>--
%
%\begin{exr} Simplifier les expressions suivantes: 
%    \begin{tasks}[label={},before-skip=1pt plus 1pt minus 1pt,after-skip=1pt,after-item-skip=1pt plus 1pt minus 1pt](3)
%    \item $A=\left( a^{-2}\right)^3\times  a$ 
%    \item $B=\left( a^{-5} b^2\right)^{-1}\times  a b^{-3}$ 
%    \item $C=\dfrac{a^5 b^{-4}}{a^{-5}b^{-2}}$  
%    \item $E=\left( -2 x^5\right)^{-4}$  
%    \item $F=-2 x^3\times  5x\times  3^{-2} x^{-5}$ 
%    \end{tasks}
%\end{exr}  
%
%--<<--<<--<<--<<--<<--<<--<<--<<--<<--<<--<<--<<--<<--<<--<<--<<--<<--<<--<<--<<--<<--<<--
%-->>-->>-->>-->>-->>-->>-->>-->>-->>-->>-->>-->>-->>-->>-->>-->>-->>-->>-->>-->>-->>-->>--
%
%\begin{exr} 
%Simplifier les expressions suivantes, où $(x;y)\in \R*\times\R*$ :
%    \begin{tasks}[label={}](5)
%    \item $A=(-a)^2(-b)^3$ 
%    \item $B=(-a)^3(-b)^5$ 
%    \item $C=(-a)^5(-b)^3$  
%    \item $D=\frac{(-a)^3}{(ab)^2}$  
%    \item $E=\frac{ab^{-2}}{a^{-3}b}$ 
%    \end{tasks}
%\end{exr}
%
%--<<--<<--<<--<<--<<--<<--<<--<<--<<--<<--<<--<<--<<--<<--<<--<<--<<--<<--<<--<<--<<--<<--
%-->>-->>-->>-->>-->>-->>-->>-->>-->>-->>-->>-->>-->>-->>-->>-->>-->>-->>-->>-->>-->>-->>--
%
%\begin{exr} 
%Calculer  $a=(-1)^{2005}+(-1)^{2006}$ et $b=(-1)^{100+((-1)^{13})^7}$.
%\end{exr}
%
%--<<--<<--<<--<<--<<--<<--<<--<<--<<--<<--<<--<<--<<--<<--<<--<<--<<--<<--<<--<<--<<--<<--
%-->>-->>-->>-->>-->>-->>-->>-->>-->>-->>-->>-->>-->>-->>-->>-->>-->>-->>-->>-->>-->>-->>--
%
%\begin{exr} 
%Écrire sous la forme $a\sqrt{3}$ avec $a \in \Q$.
%    \begin{tasks}[label={}](5)
%    \item $a=(\sqrt{3})^{-3}$
%    \item $b=(\sqrt{3})^{-13}\times (\sqrt{3})^6$
%    \end{tasks}
%\end{exr}
%
%--<<--<<--<<--<<--<<--<<--<<--<<--<<--<<--<<--<<--<<--<<--<<--<<--<<--<<--<<--<<--<<--<<--
%-->>-->>-->>-->>-->>-->>-->>-->>-->>-->>-->>-->>-->>-->>-->>-->>-->>-->>-->>-->>-->>-->>--
%
%\begin{exr} [name={gros calcul avec fractions a la calculatrice}] % collection Eureka, Dunod
%Mon arrière grand-mère (Mémé pour les intimes) va très bien. Elle a
%m\^eme gardé son horrible go\^ut pour les fractions et lorsqu'on lui
%demande son \^age, elle répond :
%\[
%\frac{2\times\left(-\frac{6}{11}\right)+\frac{\frac{6}{11}+3}{4}}{\frac{10}{816}\times\left(-\frac{6}{11}\right)+\frac{\frac{6}{11}+2}{544}}.
%\]
%Quel est son \^age ?
%% 102 ans
%\end{exr}