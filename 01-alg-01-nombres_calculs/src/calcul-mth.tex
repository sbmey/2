% !TeX root = ../../../../courant/2-02-alg-2-calculs.tex
% !TeX program = LuaLaTeX-SE
% !TeX encoding = UTF-8
% !TeX spellcheck = fr_FR
%
\begin{center}
\bfseries \Large 
Ces exercices se font sans calculatrice
\end{center}
%
%--<<--<<--<<--<<--<<--<<--<<--<<--<<--<<--<<--<<--<<--<<--<<--<<--<<--<<--<<--<<--<<--<<--
%-->>-->>-->>-->>-->>-->>-->>-->>-->>-->>-->>-->>-->>-->>-->>-->>-->>-->>-->>-->>-->>-->>--
%
\begin{mth}
    \begin{enumerate}[label=\bfseries\arabic*.]
    \item \'Ecrire sans valeur absolue $A=-4\times\abs{-2}+\abs{2\pi-7}$.
    \item Interpréter les équations en terme de distance avant de les résoudre en utilisant la droite numérique :
        \begin{enumerate-}[label=\alph*.](3)
        \item $\abs{x-5}=2$
        \item $\abs{4x-3}=9$
        \item $\abs{2x^2-4}=6$
        \end{enumerate-}
    \item Résolvez (algébriquement) les équations suivantes :
        \begin{enumerate-}[label=\alph*.](2)
        \item $\abs{4x-3}=9$
        \item $\abs{6x-1}=\abs{7-8x}$
        \end{enumerate-}
\end{enumerate}
\end{mth}
%
%--<<--<<--<<--<<--<<--<<--<<--<<--<<--<<--<<--<<--<<--<<--<<--<<--<<--<<--<<--<<--<<--<<--
%-->>-->>-->>-->>-->>-->>-->>-->>-->>-->>-->>-->>-->>-->>-->>-->>-->>-->>-->>-->>-->>-->>--
%
\begin{mth} Simplifier l'expression $A=\dfrac{3}{2x+1}+\dfrac{x+4}{3x-2}$.
\end{mth}
%
%--<<--<<--<<--<<--<<--<<--<<--<<--<<--<<--<<--<<--<<--<<--<<--<<--<<--<<--<<--<<--<<--<<--
%-->>-->>-->>-->>-->>-->>-->>-->>-->>-->>-->>-->>-->>-->>-->>-->>-->>-->>-->>-->>-->>-->>--
%
\begin{mth}
\begin{enumerate}[label=\bfseries\arabic*.]
\item Quelle est la valeur de $\sqrt{36}$ ? pourquoi ?
\item Que pensez-vous de $\sqrt{-16}$ ?
\item On sait que $(-1,2)^2=1,44$ et $1,44^2=2,0736$. Donner la valeur décimale de $\sqrt{1,44}$.
\item Calculer $\sqrt{6^2+8^2}$.
\end{enumerate}
\end{mth}
%
%
%
\begin{mth} Simplifier les nombres suivantes.

$a=\sqrt{432}$, \hfill
$b=\sqrt{20}-\sqrt{80}$, \hfill
$c=\sqrt{\frac{27}{28}}$, \hfill
$d=\dfrac{22}{7\sqrt{3}}$, \hfill
$e=\dfrac{5}{2-\sqrt3}$.
\end{mth}
%
%--<<--<<--<<--<<--<<--<<--<<--<<--<<--<<--<<--<<--<<--<<--<<--<<--<<--<<--<<--<<--<<--<<--
%-->>-->>-->>-->>-->>-->>-->>-->>-->>-->>-->>-->>-->>-->>-->>-->>-->>-->>-->>-->>-->>-->>--
%
\begin{mth}
Résoudre les équations suivantes : 
    \begin{enumerate-}(2)
    \item $-5x+3=2x+9$
    \item $(3x-7)(9x+5)=0$
    \end{enumerate-}
\end{mth}

%
%--<<--<<--<<--<<--<<--<<--<<--<<--<<--<<--<<--<<--<<--<<--<<--<<--<<--<<--<<--<<--<<--<<--
%-->>-->>-->>-->>-->>-->>-->>-->>-->>-->>-->>-->>-->>-->>-->>-->>-->>-->>-->>-->>-->>-->>--
%
\begin{mth}
Développer (et réduire !) les expressions suivantes :

\begin{tasks}[label=\labelitemi,after-item-skip=1pt](3)
\task $A(x)=(3x+4)^2$,
\task $B(x)=(3+4x)(3-4x)$,
\task  $C(x)=(4x-3)^2$,
\end{tasks}
\end{mth}
%
%--<<--<<--<<--<<--<<--<<--<<--<<--<<--<<--<<--<<--<<--<<--<<--<<--<<--<<--<<--<<--<<--<<--
%-->>-->>-->>-->>-->>-->>-->>-->>-->>-->>-->>-->>-->>-->>-->>-->>-->>-->>-->>-->>-->>-->>--
%
\begin{mth}
Factoriser les expressions suivantes :

\begin{tasks}[label=\labelitemi,after-item-skip=1pt](3)
\task $A(x)=(3x+4)^2-(3-5x)^2$,
\task $B(x)=x^2+2x+1$,
\task  $C(x)=25x^2-30x+9$,
\end{tasks}
\end{mth}
%\begin{exr} Ecrire sous forme de fraction irréductible les nombres: 
%$a=\dfrac{2}{ \dfrac{x+1}{3}}$ ;\ 
%$b=\dfrac{ \dfrac{x^2+x}{x-3}}{ \dfrac{x+1}{x^2-9}}$ ; 
%$c=2+\dfrac{ \dfrac{1}{3}x}{x+1}$
%\end{exr}
%
%--<<--<<--<<--<<--<<--<<--<<--<<--<<--<<--<<--<<--<<--<<--<<--<<--<<--<<--<<--<<--<<--<<--
%-->>-->>-->>-->>-->>-->>-->>-->>-->>-->>-->>-->>-->>-->>-->>-->>-->>-->>-->>-->>-->>-->>--
%
%\begin{exr} Ecrire les nombres suivants sous forme d'une seule fraction 
%sans radicaux au dénominateur:  
%$a=\dfrac{\sqrt3x}{\sqrt2}-\dfrac{\sqrt2x}{\sqrt3}$ ;
%$b=1-\dfrac{3}{\sqrt{x}+2}$;
%$c=\dfrac{\sqrt{x+1}+\sqrt{x}}{\sqrt{x+1}-\sqrt{x}}$
%\end{exr}
%
%--<<--<<--<<--<<--<<--<<--<<--<<--<<--<<--<<--<<--<<--<<--<<--<<--<<--<<--<<--<<--<<--<<--
%-->>-->>-->>-->>-->>-->>-->>-->>-->>-->>-->>-->>-->>-->>-->>-->>-->>-->>-->>-->>-->>-->>--
%
%\begin{exr} Simplifier les expressions suivantes: 
%    \begin{tasks}[label={},before-skip=1pt plus 1pt minus 1pt,after-skip=1pt,after-item-skip=1pt plus 1pt minus 1pt](3)
%    \item $A=\left( a^{-2}\right)^3\times  a$ 
%    \item $B=\left( a^{-5} b^2\right)^{-1}\times  a b^{-3}$ 
%    \item $C=\dfrac{a^5 b^{-4}}{a^{-5}b^{-2}}$  
%    \item $E=\left( -2 x^5\right)^{-4}$  
%    \item $F=-2 x^3\times  5x\times  3^{-2} x^{-5}$ 
%    \end{tasks}
%\end{exr}  
%
%--<<--<<--<<--<<--<<--<<--<<--<<--<<--<<--<<--<<--<<--<<--<<--<<--<<--<<--<<--<<--<<--<<--
%-->>-->>-->>-->>-->>-->>-->>-->>-->>-->>-->>-->>-->>-->>-->>-->>-->>-->>-->>-->>-->>-->>--
%
%\begin{exr} 
%Simplifier les expressions suivantes, où $(x;y)\in \R*\times\R*$ :
%    \begin{tasks}[label={}](5)
%    \item $A=(-a)^2(-b)^3$ 
%    \item $B=(-a)^3(-b)^5$ 
%    \item $C=(-a)^5(-b)^3$  
%    \item $D=\frac{(-a)^3}{(ab)^2}$  
%    \item $E=\frac{ab^{-2}}{a^{-3}b}$ 
%    \end{tasks}
%\end{exr}
%
%--<<--<<--<<--<<--<<--<<--<<--<<--<<--<<--<<--<<--<<--<<--<<--<<--<<--<<--<<--<<--<<--<<--
%-->>-->>-->>-->>-->>-->>-->>-->>-->>-->>-->>-->>-->>-->>-->>-->>-->>-->>-->>-->>-->>-->>--
%
%\begin{exr} 
%Calculer  $a=(-1)^{2005}+(-1)^{2006}$ et $b=(-1)^{100+((-1)^{13})^7}$.
%\end{exr}
%
%--<<--<<--<<--<<--<<--<<--<<--<<--<<--<<--<<--<<--<<--<<--<<--<<--<<--<<--<<--<<--<<--<<--
%-->>-->>-->>-->>-->>-->>-->>-->>-->>-->>-->>-->>-->>-->>-->>-->>-->>-->>-->>-->>-->>-->>--
%
%\begin{exr} 
%Écrire sous la forme $a\sqrt{3}$ avec $a \in \Q$.
%    \begin{tasks}[label={}](5)
%    \item $a=(\sqrt{3})^{-3}$
%    \item $b=(\sqrt{3})^{-13}\times (\sqrt{3})^6$
%    \end{tasks}
%\end{exr}
%
%--<<--<<--<<--<<--<<--<<--<<--<<--<<--<<--<<--<<--<<--<<--<<--<<--<<--<<--<<--<<--<<--<<--
%-->>-->>-->>-->>-->>-->>-->>-->>-->>-->>-->>-->>-->>-->>-->>-->>-->>-->>-->>-->>-->>-->>--
%
%\begin{exr} [name={gros calcul avec fractions a la calculatrice}] % collection Eureka, Dunod
%Mon arrière grand-mère (Mémé pour les intimes) va très bien. Elle a
%m\^eme gardé son horrible go\^ut pour les fractions et lorsqu'on lui
%demande son \^age, elle répond :
%\[
%\frac{2\times\left(-\frac{6}{11}\right)+\frac{\frac{6}{11}+3}{4}}{\frac{10}{816}\times\left(-\frac{6}{11}\right)+\frac{\frac{6}{11}+2}{544}}.
%\]
%Quel est son \^age ?
%% 102 ans
%\end{exr}