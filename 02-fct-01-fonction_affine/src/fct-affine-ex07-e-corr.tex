% !TeX root =../../../courant/2-05-fct-2-affine.tex
%
%
%
%
\begin{exr}
  \begin{itemize}
  \item Pour $x$~\euro{} de ventes et avec le contrat A :

Le salaire du commercial est de  $1800$~€ auxquels s'ajoutent $3$~\% de $x$~\euro{} soit $0,03x$~\euro{}.

Son salaire est donc $S_A(x)=1800+0,03x$. 
  \item Pour $x$~\euro{} de ventes et avec le contrat B : 

Le salaire du commercial est de  $1500$~€ auxquels s'ajoutent $5$~\% de $x$~\euro{} soit $0,05x$~\euro{}.

Son salaire est donc $S_B(x)=1500+0,05x$. 
  \item Le contrat B est le plus avantageux (pour le commercial) lorsque $S_B(x)>S_A(x)$.\newline
$S_B(x)>S_A(x) \ssi 1500+0,05x>1800+0,03x \ssi 0,02x>300 \ssi x>\frac{300}{0,02} \ssi x>15000$.

Le contrat B est donc plus avantageux pour le commercial dès que le montant de ses ventes est strictement supérieur à $15000$~€.
  \end{itemize}
\end{exr}
%
%
%
