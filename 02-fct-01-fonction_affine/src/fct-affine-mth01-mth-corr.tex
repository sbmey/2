\setcounter{exm}{0}
\begin{exm}
    \begin{itemize}
    \item 
        \begin{itemize}
        \item La taux d'accroissement de $f$ entre $1$ et $2$ est $\frac{f(2)-f(1)}{2-1}=\frac{2^2-1^2}{1}=4-1=3$.
        \item La taux d'accroissement de $f$ entre $2$ et $5$ est $\frac{f(5)-f(2)}{5-2}=\frac{5^2-2^2}{5-2}=\frac{25-4}{3}=\frac{21}{3}=7$.
        \item Les taux d'accroissements sont différents donc $f$ n'est pas une fonction affine.
        \end{itemize}
    \item Pour tous les réels $x_1$ et $x_2$ tels que $x_1\neq x_2$ : 
    
    $g(x_2)-g(x_1)=3x_2-1-(3x_1-1)=3x_2-1-3x_1+1=3(x_2-x_1)$ donc le taux d'accroissement de $g$ entre $x_1$ et $x_2$ est  $\frac{g(x_2)-g(x_1)}{x_2-x_1}=\frac{3(x_2-x_1)}{x_2-x_1}=3$.

Les taux d'accroissements de $g$ sont tous égaux à $3$ donc $g$ est une fonction affine.
    \end{itemize}
\end{exm}
