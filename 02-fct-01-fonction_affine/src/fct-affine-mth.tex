\begin{centered}
Le plan est muni d'un repère $\OIJ$.
\end{centered}
%
%
%
\begin{exr}
$f$ est la fonction affine telle que $f(1)=1$ et $f(10)=-17$. Déterminer $f(x)$.
\end{exr}
%
%
\begin{exr}
Soient $f$, $g$ et $h$ les fonctions définies par  $f(x)=2x-1$, $g(x)=-\frac34x+3$ et $h(x) = -2$.

Tracer dans un même repère leurs représentations graphiques.
\end{exr}
%
%
%
\begin{exr}
\begin{enumerate}
\item Soient les points $A\cord{20;-19}$ et $B\cord{-30;24}$. 

Ces points appartiennent-ils à la droite $(d)$ d'équation $y=-0,5x+9$ ?
\item Soient les points $C\cord{3;6}$ et $D\cord{-6;0}$. 

Déterminer l'équation réduite de la droite $\droite{CD}$ puis définir la fonction  $f$ dont la représentation graphique est $\droite{CD}$.
\end{enumerate}
\end{exr}
%
%
%
\begin{exr}
$f$, $g$ et $h$ sont les fonctions définies par $f(x)=-1,5$, $g(x)=3x-2$ et $h(x)=-5x+6$. Donner leurs variations.
\end{exr}

\begin{exr}
Résoudre les inéquations $2x-3<6$ et $4-5x\geqslant 1$.
\end{exr}
%
%
%
\begin{exr}
\begin{enumerate}
\item Dresser le tableau des signes de la fonction $f\colon x\mapsto -3(2x-5)(6-4x)$.
\item Dresser le tableau des signes de la fonction $g\colon x\mapsto \frac{-3(2x-5)}{6-4x}$.
\item Résoudre l'inéquation $f(x)\geqslant0$.
\item Résoudre l'inéquation $g(x)\geqslant0$.
\end{enumerate}
\end{exr}