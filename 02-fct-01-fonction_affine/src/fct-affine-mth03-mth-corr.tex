\setcounter{exm}{2}

\begin{exm}
\begin{enumerate}
\item 
    \begin{itemize}
    \item $-0,5x_A+9=-0,5\times 20 +9=-10+9=-1$ donc $y_A\neq -0,5x_A+9$ donc $A\not\in (d)$.
    \item $-0,5x_B+9=-0,5\times (-30) +9=15+9=24$ donc $y_B = -0,5x_B+9$ donc $B\in (d)$.
    \end{itemize}
\item $x_C \neq x_D$ donc $\droite{CD}$ n'est pas parallèle à $\Oy$ donc $\droite{CD}$ à une équation de la forme $y=ax+b$. 

Le coefficient directeur de $\droite{CD}$ est $a=\frac{y_D-y_C}{x_D-x_C}=\frac{0-6}{-6-3}=\frac{-6}{-9}=\frac{2}{3}$. L'équation réduite de $\droite{CD}$ est donc $y=\frac23x+b$.

Or  $C\cord{3;6}\in\droite{CD}$ donc $y_C=\frac23\times x_C + b$ soit $6 = \frac23 \times 3 + b$ d'où $6 = 2 + b$ puis $b = 6-2 = 4$.

L'équation réduite de $\droite{CD}$ est donc $y = \frac23 x + 4$.
\end{enumerate}
\end{exm}
