% !TeX root =../../../courant/2-05-fct-2-affine.tex
%
\begin{minipage}[t]{\linewidth-7cm}
  \begin{exr}
    \begin{enumerate}
    \item Déterminez par des lectures graphiques :
      \begin{enumerate}
      \item Une équation des droites $d_1$, $d_2$ et $d_3$.
      \item Une expression des fonctions affines $f_4$, $f_5$ et $f_6$ représentées par $d_4$, $d_5$ et $d_6$.
      \end{enumerate}
    \item Tracer les représentations graphiques $D_1$ et $D_2$ des fonctions $g_1$ et $g_2$ définies par $g_1(x)=\frac25x-4$ et $g_2(x)=-x+3$.
    \item Tracer les droites $D_3$ et $D_4$ d'équations respectives\newline $y=-1,5x+5$ et $y=\frac23x$. 
    \item Les points $A\cord{7;4,66667}$ et $B\cord{\frac16;\frac19}$ appartiennent-ils à $D_4$ ? 
%    \item Résolvez par le calcul les inéquations $-1,5x+5\geqslant 2$, $2x+1<0$ et $2x+1<-1,5x+5$. 
%    
%    Vérifiez graphiquement vos résultats.
    \end{enumerate}
  \end{exr}
\end{minipage}
\hfill
  \begin{tikzpicture}[x=0.6cm,y=0.5cm,align at top]
    \drawrepere{-5,5,-5,6}
    \path[use as bounding box] (-5,6) rectangle (5,-3);
    \draw[color=black, domain=-3:2.48, line width=1pt] plot[id=f] function{2*x+1}; 
    \node[below right,fill=white,inner sep=0pt,outer sep=1pt] at (-2,-3) {$d_1$};
    \draw[color=black, domain=-5:4 , line width=1pt] plot[id=f] function{-0.75*x-2};
    \node[below left,fill=white,inner sep=0pt,outer sep=1pt] at (3,-4.25) {$d_2$};
    \draw[color=black,line width=1pt] (-3,-5) -- (-3,6)  node[below left,fill=white,inner sep=0pt,outer sep=1pt] {$d_3$};
    \draw[color=black, domain=-5:5 , line width=1pt] plot[id=f] function{4}; 
    \node[above,fill=white,inner sep=0pt,outer sep=1pt] at (4,4) {$d_4$};
    \draw[color=black, domain=-5:5 , line width=1pt] plot[id=f] function{-0.5*x-1};
    \node[below,fill=white,inner sep=0pt,outer sep=1pt] at (4,-3){$d_5$};
    \draw[color=black, domain=-5:5 , line width=1pt] plot[id=f] function{-3.0/5*x+2};
    \node[above,fill=white,inner sep=0pt,outer sep=1pt] at (-5,5) {$d_6$};
%    \draw[color=black, domain=-0.65:5, line width=1pt] plot[id=f] function{-1.5*x+5};
%    \node[left,fill=white,inner sep=0pt,outer sep=1pt] at (-0.5,5.75) {$D_3$};
%    \draw[color=black, domain=-5:5 , line width=1pt] plot[id=f] function{2.0/3*x};
%    \node[below,fill=white,inner sep=0pt,outer sep=1pt] at (-5,-10/3) {$D_4$};
  \end{tikzpicture}
