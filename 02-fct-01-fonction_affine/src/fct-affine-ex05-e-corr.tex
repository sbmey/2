% !TeX root =../../../courant/2-05-fct-2-affine.tex
%
%
%
%
\begin{exr}
  \begin{enumerate}
  \item  
      \begin{enumerate}
      \item \setpoints{A,2,3;B,-1,-15}\typepoints\mypoints

\setdroite[id=a]{A,B}\droiteredac{a}{eq}
      \item \setpoints{A,1,17;B,15,10}\typepoints\mypoints

\setdroite[id=b]{A,B}\droiteredac{b}{eq}%
      \item \setpoints{A,-4,2;B,-4,3}\typepoints\mypoints

\setdroite[id=c]{A,B}\droiteredac{c}{eq}%%
      \end{enumerate}
  \item 
      \begin{enumerate}
      \item $f(1)=5$ et $f(7)=0$

$f$ est une fonction affine donc $f(x)=ax+b$.

Le coefficient directeur est $a=\frac{f(7)-f(1)}{7-1}=\frac{0-5}{6}=-\frac56$.

On a donc $f(x)=-\frac56x+b$.

Or $f(1)=5$ et $f(1)=-\frac56\times1+b=-\frac56+b$ donc $-\frac56+b=5$ d'où $b=5+\frac56=\frac{30}{6}+\frac56=\frac{35}{6}$.

On a donc $f(x)=-\frac56x+\frac{35}{6}$.
      \item $f(10)=17$ et $f(-4)=-11$

$f$ est une fonction affine donc $f(x)=ax+b$.

Le coefficient directeur est $a=\frac{f(10)-f(-4)}{10-(-4)}=\frac{17-(-11)}{14}=\frac{28}{14}=2$.

On a donc $f(x)=2x+b$.

Or $f(10)=17$ et $f(10)=2\times10+b=20+b$ donc $20+b=17$ d'où $b=17-20=-3$.

On a donc $f(x)=2x-3$.
      \item $f(4)=1$ et $f(16)=10$.

$f$ est une fonction affine donc $f(x)=ax+b$.

Le coefficient directeur est $a=\frac{f(16)-f(4)}{16-4}=\frac{10-1}{12}=\frac{9}{12}=0,75$.

On a donc $f(x)=0,75x+b$.

Or $f(4)=1$ et $f(4)=0,75\times4+b=3+b$ donc $3+b=1$ d'où $b=1-3=-2$.

On a donc $f(x)=0,75x-2$.
      \end{enumerate}
 \end{enumerate}
\end{exr}
%
%
%
