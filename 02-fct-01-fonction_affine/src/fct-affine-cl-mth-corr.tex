%
%----------------------------------------------
%
\section{Les fonctions affines}
%
%----------------------------------------------
%
\subsection{Définitions et caractérisation}
%
%
%
\begin{exr}
$f$ est une fonction affine donc $f(x)=mx+p$.
    \begin{itemize}
    \item Le coefficient directeur de $f$ est $m=\frac{f(10)-f(1)}{10-1}=\frac{-17-1}{9}=\frac{-18}{9}=-2$. 
    On a donc $f(x)=-2x+p$.
    \item $f(x)=-2x+p$ donc $f(1)=-2\times1+p=-2+p$ or $f(1)=1$ donc $-2+p=1$ d'où $p=1+2=3$.
    \item On a donc $f(x)=-2x+3$.
    \end{itemize}
\end{exr}
%
%----------------------------------------------
%
\subsection{Représentation graphique}
%
%----------------------------------------------
%
\begin{exr}
Soient $f$, $g$ et $h$ les fonctions définies par  $f(x)=2x-1$, $g(x)=-\frac34x+3$ et $h(x) = -2$.

Pour tracer les droites représentant $f$, $g$ et $h$ :
    \begin{itemize}
    \item on place le point correspondant à l'ordonnée à l'origine,
    \item on trace, à partir du point précédent, le triangle  représentant le coefficient directeur (en rouge sur le graphique),
    \item il ne reste qu'à tracer la droite !
    \end{itemize}
    \begin{tikzpicture}
    \drawrepere{-3,7,-4,5}
%
    \draw[line width=1.5pt] plot[domain=-3:7] function {2*x-1};
    \node[below right] at (3,5) {$\crbf$};
    \draw[fill,red] (0,-1) circle (2pt);
    \draw[fill,red] (1,1) circle (2pt);
    \draw[red,densely dashed,line width=1.5pt] (0,-1) -| (1,1) node [pos=0.25,anchor=north] {$1$} node[pos=0.75,anchor = south west] {$2$};
%
    \draw[line width=1.5pt] plot[domain=-3:7] function {-0.75*x+3};
    \node[below left] at (-1,3.75) {$\crbf{g}$};
    \draw[fill,red] (0,3) circle (2pt);
    \draw[fill,red] (4,0) circle (2pt);
    \draw[red,densely dashed,line width=1.5pt] (0,3) -| (4,0) node [pos=0.25,anchor=south east] {$4$} node[pos=0.75,anchor = west] {$-3$};
%
    \draw[line width=1.5pt] plot[domain=-3:7] function {-2};
    \node[below] at (5,-2) {$\crbf{h}$};
    \draw[fill,red] (0,-2) circle (2pt);
    \end{tikzpicture}
    
    \begin{rmq}
        \begin{itemize}
        \item Le coefficient directeur de $f$ est $2$. \'A partir d'un point de $\crbf$, on avance d'une unité puis on monte de $2$ unités pour obtenir un deuxième point de la droite.
        \item Le coefficient directeur de $g$ est $-\frac34$. \'A partir d'un point de $\crbf{g}$, on avance de $4$ unités puis on descend de $3$ unités pour obtenir un deuxième point de la droite.
        \end{itemize}
    \end{rmq}
\end{exr}
%
%
%
\begin{exr}
\begin{enumerate}
\item 
    \begin{itemize}
    \item $-0,5x_A+9=-0,5\times 20 +9=-10+9=-1$ donc $y_A\neq -0,5x_A+9$ donc $A\not\in (d)$.
    \item $-0,5x_B+9=-0,5\times (-30) +9=15+9=24$ donc $y_B = -0,5x_B+9$ donc $B\in (d)$.
    \end{itemize}
\item $x_C \neq x_D$ donc $\droite{CD}$ n'est pas parallèle à $\Oy$ donc $\droite{CD}$ à une équation de la forme $y=ax+b$. 

Le coefficient directeur de $\droite{CD}$ est $a=\frac{y_D-y_C}{x_D-x_C}=\frac{0-6}{-6-3}=\frac{-6}{-9}=\frac{2}{3}$. L'équation réduite de $\droite{CD}$ est donc $y=\frac23x+b$.

Or  $C\cord{3;6}\in\droite{CD}$ donc $y_C=\frac23\times x_C + b$ soit $6 = \frac23 \times 3 + b$ d'où $6 = 2 + b$ puis $b = 6-2 = 4$.

L'équation réduite de $\droite{CD}$ est donc $y = \frac23 x + 4$.

$\droite{CD}$ est donc la représentation graphique de la fonction affine $f$ définie par $f(x)=\frac23x+4$.
\end{enumerate}
\end{exr}
\endinput
%
%----------------------------------------------
%
\section{\'Etude d'une fonction affine}
%
%----------------------------------------------
%
%
\setcounter{subsection}{1}
%
%
\subsection{Variations}
%
%
%
\begin{exr}
\begin{itemize}
\item Le coefficient directeur de $f$ est $0$ donc $f$ est constante.
\item Le coefficient directeur de $g$ est $m=3$, il est strictement positif  donc $g$ est strictement croissante.
\item Le coefficient directeur de $h$ est $m=-5$, il est strictement négatif  donc $h$ est strictement décroissante.
\end{itemize}
\end{exr}
%
%
%
\begin{exr}
    \begin{itemize}
    \item $2x-3<6 \ssi 2x-3+3<6+3 \ssi 2x<9 \ssi \frac{2x}{2}<\frac{9}{2} \ssi x<\frac92$ ( $2>0$ donc lorsqu'on divise par $2$, on conserve le sens de l'inégalité).
    
    L'ensemble des solutions est donc $\intervalleo{-\infty;4,5}$.
    \item $4-5x\geqslant 1 \ssi 4-5x-4\geqslant 1-4\ssi -5x \geqslant -3 \ssi \frac{-5x}{-5}\leqslant \frac{-3}{-5}\ssi x\leqslant \frac35$ ($-5<0$ donc lorsqu'on divise par $-5$ on renverse le sens de l'inégalité).
    
    L'ensemble des solutions est donc $\intervalleof*{-\infty;\frac35}$.
    \end{itemize}
\end{exr}
%
%
%
\subsection{Signes}
%
%
%
\begin{exr}
\begin{enumerate}
\item Le tableau des signes de $f\colon x\mapsto -3(2x-5)(6-4x)$ est  : \hfill
    \begin{tikzpicture}[baseline={($(current bounding box.north)-(0,\ht\strutbox)$)}]
    \tkzTabInit[lgt=1.5,espcl=2]{$x$/0.5,$-3$/0.5,$2x-5$/0.5,$6-4x$/0.5,$f(x)$/0.5}{$-\infty$,$\numprint{1.5}$,$\numprint{2.5}$,$+\infty$}
    \tkzTabLine{,-,t,-,t,-,}
    \tkzTabLine{,-,t,-,z,+,}
    \tkzTabLine{,+,z,-,t,-,}
    \tkzTabLine{,+,z,-,z,+,}
    \end{tikzpicture}
\item Le tableau des signes de $g\colon x\mapsto \frac{-3(2x-5)}{6-4x}$ est :\hfill
    \begin{tikzpicture}[baseline={($(current bounding box.north)-(0,\ht\strutbox)$)}]
    \tkzTabInit[lgt=1.5,espcl=2]{$x$/0.5,$-3$/0.5,$2x-5$/0.5,$6-4x$/0.5,$f(x)$/0.5}{$-\infty$,$\numprint{1.5}$,$\numprint{2.5}$,$+\infty$}
    \tkzTabLine{,-,t,-,t,-,}
    \tkzTabLine{,-,t,-,z,+,}
    \tkzTabLine{,+,z,-,t,-,}
    \tkzTabLine{,+,d,-,z,+,}
    \end{tikzpicture}
\item L'ensemble des solutions de l'inéquation $f(x)\geqslant0$ est $\intervalleof{-\infty;1,5}\cup\intervallefo{2,5;+\infty}$.
\item L'ensemble des solutions de l'inéquation $g(x)\geqslant0$ est $\intervalleo{-\infty;1,5}\cup\intervallefo{2,5;+\infty}$.
\end{enumerate}
\end{exr}