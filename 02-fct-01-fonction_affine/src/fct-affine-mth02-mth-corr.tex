\setcounter{exm}{1}

\begin{exm}
$f$ est une fonction affine donc $f(x)=ax+b$.
    \begin{itemize}
    \item Le coefficient directeur de $f$ est $a=\frac{f(10)-f(1)}{10-1}=\frac{-17-1}{9}=\frac{-18}{9}=-2$. 
    On a donc $f(x)=-2x+b$.
    \item $f(x)=-2x+b$ donc $f(1)=-2\times1+b=-2+b$ or $f(1)=1$ donc $-2+b=1$ d'où $b=1+2=3$.
    \item On a donc $f(x)=-2x+3$.
    \end{itemize}
\end{exm}
